\documentclass{article}
\usepackage{fullpage}
\usepackage{array}
\usepackage[czech]{babel}
\usepackage{amsmath}
\usepackage{multicol}
\usepackage{amsfonts}
\usepackage{caption}
\usepackage{wrapfig}
\usepackage{graphicx}
\usepackage{caption}
\usepackage{tikzit}
\input{styly.tikzstyles}

\begin{document}
\section{Kinematika}
\begin{multicols}{2}
\noindent$a=\frac{\Delta v}{t}$\\
$s=v_0t+\frac{\Delta v t}{2}$\\
$v=g\cdot t=a\cdot t$\\
$s=\frac{1}{2}at^2 \dots h=\frac{1}{2}gt^2$\\
$v=\omega \cdot r$\\
$s=\varphi \cdot r$\\
$a_D=\frac{v^2}{r}=r\cdot\omega^2$\\
$\varepsilon = \frac{\Delta \omega}{t}$\\
$\omega=\omega_0+\varepsilon\cdot t$\\
$\varphi=\omega_0t+\frac{1}{2}\varepsilon t^2$
\end{multicols}

\section{Dynamika}
\begin{multicols}{2}
\noindent $F=m\cdot a$\\
$p=m\cdot v$ $\left[kg\cdot m\cdot s^{-1}\right]$\\
$F=\frac{\Delta p}{t}$\\
$F_T=F_N\cdot f$
\end{multicols}

\section{Práce, výkon, energie}
\begin{multicols}{2}
\noindent $W=\vec{F}\cdot\vec{s}=F\cdot s \cdot \cos{\alpha}$ $[J]$\\
$E_p=mgh$\\
$E_k=\frac{1}{2}mv^2$\\
$P_p=\frac{W}{t}$ $[W]$ (výkon)\\
$P=F\cdot v$ (okamžitý výkon)\\
$P_0=\frac{\Delta E}{\Delta t}$ (příkon)\\
$\eta=\frac{P}{P_0}$ (účinnost)\\
\end{multicols}

\noindent Dokonale pružná srážka:
\begin{multicols}{2}
\noindent $V_1=v_1\cdot \frac{m_1-m_2}{m_1+m_2}+v_2\cdot \frac{2m_2}{m_1+m_2}$\\
$V_2=v_2\cdot \frac{m_2-m_1}{m_1+m_2}+v_1\cdot \frac{2m_1}{m_1+m_2}$
\end{multicols}
\noindent Pozn. Dokonale nepružná srážka -- platí zákon zachování hybnosti.

\section{Radiální gravitační pole}
\begin{multicols}{2}
\noindent $F_g=G\frac{m_1m_2}{r^2}$\\
$\vec{K}=\frac{\vec{F_g}}{m}$ (intenzita grav. pole)\\
$\frac{T^2}{a^3} = \text{konst}$\\
$v^2=G\cdot \frac{M}{r}$\\
$\frac{4\pi ^2}{GM}=\frac{T^2}{r^3}$\\
$v_I = \sqrt{\frac{GM}{r}}$\\
$v_{II} = \sqrt{2}\cdot v_I$\\
$E_p = -G\frac{Mm}{r}$\\
$G=6,67\cdot 10^{-11}$\\
\end{multicols}

\section{Vrhy v homogenním gravitačním poli}
\begin{multicols}{2}
\noindent Osa x:\\
$v_{0x}=\cos{\alpha}\cdot v_0$\\
$v_x=v_{0x}$\\
$x=v_{0x}t$\\
Osa y:\\
$v_{0y}=\sin{\alpha}\cdot v_0$\\
$v_y=v_{0y}-gt$\\
$y=v_{0y}t-\frac{1}{2}gt^2$
\end{multicols}

\section{Tuhé těleso}
\begin{multicols}{2}
\noindent $M = F \cdot a \cdot \sin{\alpha}$ $[Nm]$\\
$E_r = \frac{1}{2}J\omega ^2$\\
$J_0$: obruč: $mr^2$, koule: $\frac{2}{5}mr^2$, válec: $\frac{1}{2}mr^2$, tyč: $\frac{1}{12}ml^2$\\
$J=J_0+md^2$
\end{multicols}
\pagebreak

\section{Struktura a vlastnosti látek}
\begin{multicols}{2}
\noindent $A_r = \frac{m_a}{u}$\\
$u = 1.66 \cdot 10^{-27} \textrm{ kg}$\\
$M_r = \frac{m_m}{u}$\\
$N_A = 6,022 \cdot 10^{23} \textrm{ mol} ^{-1}$\\
$n = \frac{N}{N_A}$ $[\textrm{mol}]$\\
$M_m = \frac{m}{n}$ $\left[\textrm{kg}\cdot \textrm{mol} ^{-1}}\right]$\\
$M_m = 10^{-3} \cdot M_r$\\
$V_m = \frac{V}{n}$ $\left[\textrm{m}^3\cdot\textrm{mol}^{-1}\right]$\\
$\rho = \frac{M_m}{V_m}$
\end{multicols}

\section{Termodynamika}
\begin{multicols}{2}
\noindent $\Delta U = Q + W$\\
$Q = \frac{S\cdot \Delta t \cdot \lambda}{d}\cdot \tau$\\
$C = \frac{Q}{\Delta t}$ $[J\cdotK^{-1}]$\\
$c = \frac{C}{m}$\\
$C_m = \frac{Q}{u\cdot \Delta t}$\\
$Q = mc\Delta t$\\
$\Delta l = l_0 \alpha \Delta t$\\
$l = l_0(\alpha\Delta t + 1)$\\
$\Delta V = V_0 \beta \Delta t$\\
$V = V_0(\beta\Delta t + 1)$
$\beta = 3\alpha$ \\
$\rho = \rho _0 (1 - \beta \Delta t)$
\end{multicols}


\section{Struktura a vlastnosti plynů}
\begin{multicols}{2}
\noindent$p=\frac{1}{3}\rho v^2$\\
$E=\frac{i}{2}kT$, kde $k=1,38\cdot 10^{-23}$ $JK^{-1}$\\
$v=\sqrt{\frac{ikT}{m_0}}$, pro pohyb $i=3$\\
$pV=NkT=RnT$, tj. $\frac{pV}{T}=\textrm{konst}$\\
$R = 8,31$ $J\cdot mol ^{-1}K^{-1}$\\
$Q=\Delta U+W^\prime$\\
$\Delta U=\frac{i}{2}nR\Delta T$
\end{multicols}
\begin{multicols}{}
    \begin{description}
        \vspace{-0.5em}
        \setlength\itemsep{0.15em}
        \item[$i.$] izotermický: $T=\textrm{konst}$ a $Q=W^\prime$
        \item[$ii.$] izochorický: $V=\textrm{konst}$ a $Q=\Delta U$
        \item[$iii.$] izobarický: $p=\textrm{konst}$ a $W^\prime = p\cdot \Delta V$
        \item[$iv.$] adiabatický: $Q=0$ a $p\cdot V^\kappa = \textrm{konst}$, kde $\kappa = 1+\frac{2}{i}$
    \end{description}
\end{multicols}

\section{Mechanika tekutin}
\begin{multicols}{2}
\noindent $W=Fx$\\
$p=h\rho g$\\
$F_V=V\rho g$\\
$Q_V=\frac{V}{t}$\\
$S_1v_1=S_2v_2$\\
$E_T=p\Delta V$\\
$\rho gh +\frac{1}{2}\rho v^2+p=\mathrm{konst}  $\\
$ h = \textrm{konst}\Rightarrow \frac{1}{2}\rho v^2 + p=\textrm{konst}$\\
$v=\sqrt{2hg}$\\
$d=2\sqrt{h\cdot h^\prime}$\\
$F_{ODP}=\frac{1}{2}CS\rho v^2$, kde $\rho$ je prostředí
\end{multicols}
Hodnoty součinitele odporu $C$ pro vybraná tělesa:
\begin{center}
\begin{tabular}{ c c c c c}
 \scalebox{1}{\tikzfig{miska}}  & \scalebox{1}{\tikzfig{obr_miska}} & \scalebox{1}{\tikzfig{koule}} & \scalebox{1}{\tikzfig{cara}} & \scalebox{0.9}{\tikzfig{kridlo}}\\\\
 1,33 & 0,34 & \hspace{0.4cm}0,48 & \hspace{0.4cm}1,12 & 0,03
\end{tabular}
\end{center}



\end{document}
